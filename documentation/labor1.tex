\documentclass[12pt]{report}

\usepackage[a4paper,
			inner = 35mm,
			outer = 25mm,
			top = 25mm,
			bottom = 25mm]{geometry}
\usepackage{lmodern}
\usepackage[magyar]{babel}
\usepackage[utf8]{inputenc}
\usepackage[T1]{fontenc}
\usepackage[hidelinks]{hyperref}
\usepackage{graphicx}
\usepackage{amssymb}
\usepackage{setspace}
\usepackage[nottoc,numbib]{tocbibind}
\usepackage{amsthm}

% \setcounter{secnumdepth}{3}
\onehalfspacing

\newtheorem{mydef}{Definició}
\newtheorem{mytetel}{Tétel}

\begin{document}

\begin{titlepage}
	\begin{center}
		\vspace*{1cm}
		
		\textbf{\LARGE 
			Forgalom igény tudatos hálózat tervezés minimális torlódással és úthosszal
		}
	
	
		\vspace{0.5cm}
	
		\textbf{\normalsize Tudáskezelő rendszerek II. labor összefoglaló}
		
		\vfill
		
		\Large Szecsődi Imre
		
		\vspace{2.8cm}
		
		\the\year
		
	\end{center}
\end{titlepage}

\tableofcontents
	
\chapter{Bevezetés}

A labor munka a Demand-Aware Network Design with Minimal Congestion and Route Lengths \cite{avin_demand-aware_nodate} cikk alapján készült.

\section{Motiváció}

\begin{itemize}
	\item A technika előrehaladásával egyre nagyobb lett a feldolgozandó adatok mennyisége
	\item Adattárházakban a szerverek közötti kommunikáció is ezáltal megnövekedett
	\item A jelenlegi hálózatok a legrosszabb esetre vannak tervezve, azaz, hogy majdnem teljes sávszélességű, kétirányú kapcsolat álljon fent bármelyik két szerver között
	\item A valós kommunikáció nem ezt a sémát követi, hanem túlnyomó részt megadott párok között történik a legtöbb kommunikáció
\end{itemize}

Microsoft Research ProjecToR \cite{ghobadi_projector:_2016}.

\begin{itemize}
	\item Nézzünk meg pár valós példát, Microsoft adattárházában  250 ezer szervert 5 production klaszterben elosztva
\end{itemize}

\subsection{Hálózat tervezési stratégiák}

\begin{itemize}
	\item A technika fejlődésével elérhetővé váltak eszközök arra, hogy egy adott hálózatot újra konfiguráljunk, attól függően milyen terhelés éri
	\begin{itemize}
		\item pl, korábbi kommunikációs minták alapján
	\end{itemize}
	\item Két fő optimalizációs lehetőség van, legyen rövid az út (a) vagy legyen minimális a torlódás (b)
	\item A cikk bemutat egy módszert arra, hogy lehet mindkettőre majdnem optimális megoldást adni egyszerre (c)
\end{itemize}


\subsection{Adattárházak hálózati felépítése}

\begin{itemize}
	\item Core switch
	\item Aggregation Swtiches
	\item Top of Rack Switches
	\item In-Rack Switches
\end{itemize}

\subsection{Újrakonfigurálás megvalósítása}

\begin{itemize}
	\item Átlag hálózatok statikusan vannak konfigurálva, nem  sok lehetőséget adva annak, hogy változtassunk 
	\begin{itemize}
		\item pl. Ethernet switchek
	\end{itemize}
	\item Optikai switchek már újra tudják konfigurálni magukat, de ezek "lassúak"
	\item Microsoft Research - ProjecToR\cite{ghobadi_projector:_2016}, lézer segítségével kiváltani az optikai swticheket
	\begin{itemize}
		\item 12 $\mu s$ váltás idő ( 2500x gyorsabb mint egy optikai hálózati switch)
	\end{itemize}
	
\end{itemize}

\section{Labor célja}

A labor célja a cikkben\cite{avin_demand-aware_nodate} bemutatott algoritmus implementálása, és annak alkalmazása különböző véletlenszerűen generált gráfokra. 
A kapott eredményeket össze lehet hasonlítani a megadott elméleti korlátokkal.

\section{Laborban megvalósított munka}

A labor ideje alatt elkészült egy keretrendszer, ami segítségével tesztelhető a szerzők által felvázolt algoritmus. 
A keretrendszer Python \cite{noauthor_python_nodate} nyelven íródott.
Egy véletlen gráfok generálására egy külső csomag lett használva \cite{noauthor_networkx_nodate}

\chapter{Modell}

\section{Forgalom igény tudatos hálózat tervezés probléma}

\begin{itemize}
	\item Vegyünk egy hálózatot meghatározott számú csomóponttal
	\item A hálózathoz tartozik egy demand mátrix, ami leírja a valószínűségét annak, hogy $i$ forrásból mekkora eséllyel lesz adat küldve $j$ célba
	\item A cél, hogy ezen adatból egy olyan hálózati séma készítése, ami kis torlódást és rövid utakat eredményez, ez mellett még skálázható is
\end{itemize}

\section{Formális felírás}

\begin{itemize}
	\item Adott $N$ darab csúcspont  $V = \{1, ..., N\}$, és egy kommunikációs séma $M_D$ ami egy $N\times N$ mátrix
	\item A mátrix $(i, j)$ eleméhez tartozik egy $p(i, j)$ valószínűség, ahol $i$ a forrás csomópont és $j$ a cél
	\item A bemeneti mátrix ábrázolható egy irányított
	$G_D$ gráfban, ahol az élsúlyok a két pont közötti kommunikációs valószínűség
	\item Az algoritmus feltétele, hogy a mátrix ritka legyen
	\item Egy $N$ hálózatra a torlódást és az úthosszt útválasztási sémával fogjuk definiálni
	\item Egy útválasztási séma az $N$ hálózatra $\Gamma(N)$, ami $\Gamma_{uv}$ utak halmaza, ahol $(u, v)$ párok különböző utakat jelölnek
	\item $\Gamma_{uv}$ egy útsorozat, ami összeköti az $u$ pontot $v$ ponttal
\end{itemize}

\subsection{Torlódás}

\begin{mydef}
	A torlós jó
\end{mydef}

\subsection{Úthossz}

\begin{mydef}
	Az úthossz is jó
\end{mydef}

\subsection{Skálázhatóság}

\begin{mytetel}
	Itt a kapitány tétel, vétel
\end{mytetel}

\section{EgoTree}

\section{cl-DAN algoritmus}

\chapter{Megvalósítás}

\chapter{Teszt eredmények}

\chapter{Összefoglalás}

\bibliographystyle{abbrv}
\bibliography{refrences}

	
\end{document}